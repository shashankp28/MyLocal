\documentclass{beamer}
\usetheme{Warsaw}
\usepackage{amsmath}
\usepackage{amssymb}
\usefonttheme{serif}
\title{Assignment-2}
\subtitle{Software Systems Lab}
\author{Shashank P}
\institute{IIT Dharwad \\ \url{https://www.iitdh.ac.in/}}
\date{August 11, 2021}
\titlegraphic{\includegraphics[width=2.7cm, height=2.3cm]{image 2.jpeg}}
\begin{document}
	\frame{\titlepage}
	\begin{frame}
		\frametitle{Dynamic Programming}
		\begin{itemize}
			\item Characteristics of Dynamic Programming
			\begin{enumerate}
				\item \textit{Overlapping Sub-problems}
				\begin{block}{1}
					Subproblems are smaller versions of the original problem. Any
					problem has overlapping sub-problems if finding its solution involves
					solving the same subproblem multiple times.
				\end{block}
				\item \textit{Optimal Substructure}
				\begin{block}{2}
					Any problem has optimal substructure property if its overall optimal
					solution can be constructed from the optimal solutions of its
					subproblems.
				\end{block}
			\end{enumerate}
		\end{itemize}
	\end{frame}
	\begin{frame}
		\frametitle{DP Methods}
		\begin{itemize}
			\item<1-> \textbf{Top-down with Memoization}
			\begin{block}{1}
				In this approach, we try to solve the bigger problem by
				recursively finding the solution to smaller sub-problems.
				Whenever we solve a sub-problem, we cache its result so that
				we don't end up solving it repeatedly if it's called multiple
				times. Instead, we can just return the saved result.
			\end{block}
			\item<2-> \textbf{Bottom-up with Tabulation}
			\begin{alertblock}{2}
				Tabulation is the opposite of the top-down approach and avoids
				recursion. In this approach, we solve the problem ``bottom-up"
				(i.e. by solving all the related sub-problems first).
			\end{alertblock}
		\end{itemize}
	\end{frame}
	\begin{frame}
		\transsplitverticalin
		\label{dac}
		\frametitle{Algorithms}
		\begin{itemize}
			\setbeamercovered{transparent}
			\item<1-> Divide and conquer
			\item<2-> Greedy Algorithm
			\item<3-> Dynamic Programming
		\end{itemize}
	\end{frame}
	\begin{frame}
		\frametitle{Divide and Conquer}
		\only<1>{Example: \\
			\textbf{Quick-Sort: The average case run time of quick sort is}
			$O(n*log\,n)$.\textbf{This case happens when we don't exactly
				get evenly balanced partitions.}}
		\only<2>{\alert{Example:} \\
			Merge-Sort: \textcolor{orange}{The time complexity of Merge Sort is $O(n*log\,n).$
				Merge Sort is useful for sorting linked lists in $O(n*log\,n)$ time.}}
	\end{frame}
	\begin{frame}
		\setbeamercovered{dynamic}
		\frametitle{Hyperlinks}
		\begin{itemize}
			\item \hyperlink{dac}{Divide and Conquer}
			\item \hyperlink{dac}{\beamergotobutton{Greedy Algorithm}}
			\item \hyperlink{dac}{\beamerskipbutton{Dynamic Programming}}
		\end{itemize}
	\end{frame}
	\begin{frame}
		\frametitle{List of Data Structures}
		\setbeamercovered{transparent}
		\begin{itemize}
			\item<1-> Primitive
			\item<2-> Non-Primitive \newline
		\end{itemize}
			\begin{itemize}
				\setlength{\itemindent}{.3in}
				\item<3-> \textit{Linear}
				\begin{itemize}
					\setlength{\itemindent}{.3in}
					\item<4-> \textbf{Static}
					\begin{enumerate}
						\setlength{\itemindent}{.1in}
						\item<5-> Array \newline
					\end{enumerate}
					\setlength{\itemindent}{.1in}
					\item<4-> \textbf{Dynamic}
					\begin{enumerate}
						\setlength{\itemindent}{.1in}
						\item<5-> Linked List
						\item<5-> Stack
						\item<5-> Queue \newline
					\end{enumerate}
				\end{itemize}
				\item<6-> \textit{Non-Linear}
				\begin{enumerate}
					\setlength{\itemindent}{.1in}
					\item<6-> Tree
					\item<6-> Graph
				\end{enumerate}
			\end{itemize}
	\end{frame}
	\begin{frame}
		\frametitle{Data Structures}
		\begin{figure}
			\includegraphics[scale=0.28]{image 1.jpeg}
			\caption{1}
		\end{figure}
	\end{frame}
	\begin{frame}
		\begin{center}
			\begin{table}
				\begin{tabular}{|l|c|c|c|}
					\hline
					Algorithm & Best Case & Average Case & Worst Case \\ 
					\hline \hline
					Linear Search & $O(1)$ & $O(n)$ & $O(n)$ \\
					Binary Search & $O(1)$ & $O(log\,n)$ & $O(log\,n)$ \\
					Bubble sort & $O(n)$ & $O(n^2)$ & $O(n^2)$ \\ 
					Selection sort & $O(n^2)$ & $O(n^2)$ & $O(n^2)$ \\
					\hline    
				\end{tabular}
				\caption{1}
			\end{table}
		\end{center}
		\begin{theorem}[Trigonometric Identity]
			$Sin^2 \theta + Cos^2 \theta = 1$
		\end{theorem}
	\end{frame}
	\begin{frame}
		\begin{theorem}
			Let a, b, c be lengths of right angled triangle.\\
			\textbf{By definition}
			$$sin \theta = b/c \, \left(\frac{oppositeside}{hypotenuse}\right)$$
			$$cos \theta = a/c \, \left(\frac{adjacentside}{hypotenuse}\right)$$
			$sin^2 \theta + cos^2 \theta = \frac{b^2}{c^2} + \frac{a^2}{c^2} = \frac{a^2+b^2}{c^2}$ \newline \newline
			\textbf{From Pythagoras theorem} \\ \vspace{0.4cm}
			$c^2 = a^2 + b^2$ \newline \newline
			$\frac{a^2+b^2}{c^2} = 1 \implies sin^2 \theta + cos^2 \theta = 1$ \newline \newline
			\textbf{Hence Proved.}
		\end{theorem}
	\end{frame}
	\begin{frame}
		\frametitle{Multi-line equations}
		\begin{multline*}
			f(x) = x^6 + 7x^3y + 50x^3y^2 + 12x^2y^4\\ 
			- 19x^5y^4 - 10x^7y^6 + 7y^6 - m^3n^3 \notag
		\end{multline*}
		\begin{equation}
			\begin{split}
				\rho \Delta x \Delta y \Delta z \Delta \tau \partial_t c_i(t,x,\tau) &= \rho \Delta x \Delta y \Delta z \Delta \tau (p_i - d_i) \\ 
				&- \rho \Delta y, \Delta z \Delta \tau [q_{i,x}(t,x + \Delta x/2, y, z, \tau ) \\
				& \quad \quad - q_{i,x}(t,x - \Delta x/2, y, z, \tau )] \\
				&- \rho \Delta x, \Delta z \Delta \tau [q_{i,y}(t,x, y + \Delta y/2, y , z, \tau ) \\
				& \quad \quad- q_{i,y}(t,x , y - \Delta y/2,z,  z, \tau )] \\
				&- \rho \Delta x \Delta y \Delta \tau [q_{i,z}(t,x, y, z + \Delta z/2, \tau ) \\
				& \quad \quad -q_{i,z}(t, x, y, z - \Delta z/2, \tau )] \notag
			\end{split}
		\end{equation}
	\end{frame}
\end{document}